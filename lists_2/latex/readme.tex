%%%%%%%%%%%%%%%%%%%%%%%%%%%%%%%%%%%%%%%%%
% Short Sectioned Assignment
% LaTeX Template
% Version 1.0 (5/5/12)
%
% This template has been downloaded from:
% http://www.LaTeXTemplates.com
%
% Original author:
% Frits Wenneker (http://www.howtotex.com)
%
% License:
% CC BY-NC-SA 3.0 (http://creativecommons.org/licenses/by-nc-sa/3.0/)
%
%%%%%%%%%%%%%%%%%%%%%%%%%%%%%%%%%%%%%%%%%

%----------------------------------------------------------------------------------------
%	PACKAGES AND OTHER DOCUMENT CONFIGURATIONS
%----------------------------------------------------------------------------------------

\documentclass[paper=a4, fontsize=11pt]{scrartcl} % A4 paper and 11pt font size

\usepackage[T1]{fontenc} % Use 8-bit encoding that has 256 glyphs
\usepackage{fourier} % Use the Adobe Utopia font for the document - comment this line to return to the LaTeX default
\usepackage[spanish]{babel} % English language/hyphenation
\usepackage[utf8]{inputenc}
\usepackage{sectsty} % Allows customizing section commands
\allsectionsfont{\centering \normalfont\scshape} % Make all sections centered, the default font and small caps

\usepackage{fancyhdr} % Custom headers and footers
\pagestyle{fancyplain} % Makes all pages in the document conform to the custom headers and footers
\fancyhead{} % No page header - if you want one, create it in the same way as the footers below
\fancyfoot[L]{} % Empty left footer
\fancyfoot[C]{} % Empty center footer
\fancyfoot[R]{\thepage} % Page numbering for right footer
\renewcommand{\headrulewidth}{0pt} % Remove header underlines
\renewcommand{\footrulewidth}{0pt} % Remove footer underlines
\setlength{\headheight}{13.6pt} % Customize the height of the header

\setlength\parindent{0pt} % Removes all indentation from paragraphs - comment this line for an assignment with lots of text

%----------------------------------------------------------------------------------------
%	TITLE SECTION
%----------------------------------------------------------------------------------------

\newcommand{\horrule}[1]{\rule{\linewidth}{#1}} % Create horizontal rule command with 1 argument of height

\title{	
\normalfont \normalsize 
\textsc{Centro de INvestigacion en Matematicas, Maestria en Ciencias de la Computacion} \\ [25pt] % Your university, school and/or department name(s)
\horrule{0.5pt} \\[0.4cm] % Thin top horizontal rule
\huge Generacion de un Makefile \\ % The assignment title
\horrule{2pt} \\[0.5cm] % Thick bottom horizontal rule
}

\author{Juan Gerardo Fuentes Almeida} % Your name

\date{\normalsize\today} % Today's date or a custom date

\begin{document}

\maketitle % Print the title

%----------------------------------------------------------------------------------------
%	PROBLEM 1
%----------------------------------------------------------------------------------------

\section{Implementacion}

Este proyecto consiste en el diseño de un programa que realiza varias funciones entre listas ligadas, tales como inserción y eliminación de nodos, conteo de datos y concatenación de información. En primera instancia, se introduce una cadena de datos alfabéticos o alfanuméricos, y el programa guarda cada pieza de información en un nodo de una lista ligada, después se realizan varias operaciones entre los nodos y el contenido de estos, como el conteo de etiquetas e impresión del conjunto de datos en cadena.\\

En conjunción con esto se ha realizado un Makefile, en el cual están contenidas las directivas de compilación y linkeo, con opciones de optimización y debugeo.\\


\subsection{Funciones del Makefile}
\begin{itemize}
	\item make: compilación normal, genera archivos objeto en la carpeta /obj y un ejecutable en la carpeta /bin.
	\item make help: despliega archivos que se compilaras y opciones de compilación
	\item make optimized: compilación con banderas de optimización
	\item make debugging: compilaciones con opciones de debugeo
	\item make clean: borra los archivos de compilación, ejecutable y documentación
	
\end{itemize}

%------------------------------------------------



\end{document}